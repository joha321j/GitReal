\subsection{Business Case}
\subsubsection{Den nuværende situation}
\subparagraph{Timeregistrering til regning}

    Hver fredag udfylder svendene en papirsskabelon, hvor de skriver stedet arbejdet er blevet udført, hvilken dag arbejdet er udført, hvad der er gjort, hvilke materialer der er taget fra firmaets eget værksted, og hvor mange timer der er brugt.
    Dette gøres for alle timerne i ugen.
    
    Hver tirsdag eller onsdag samler sekretæren alle deres registreringer og skriver regninger til Halvorsen's kunder.

\subparagraph{Timeregistrering til aflønning}
    Hver anden fredag udfylder svendene også en anden papirsskabelon, hvor de skriver alle 14 dage, og hvor mange timer de har arbejdet de enkelte dage. Overarbejdstimer skrives separat.
    
    For både regning og aflønningstimeregistreringen er svendene derfor nød til at køre tilbage på værkstedet for at udfylde papirerne.
    
\subparagraph{Intern materialebestilling}
    Når en svend opdager, at der mangler materiale på en byggeplads, som de skal bruge, ringer de hjem til tømmermesteren, hvor de fortæller, hvad de skal bruge, samt hvor og hvornår det skal bruges. Derfor skal Torben selv huske på alle bestillingerne.
    
\subsubsection{Formålet med projektet}

    Projektet skal effektivisere den interne kommunikation i     firmaet.
    Dette gøres ved at arbejde med to problematikker.
    
    Først skal projektet sørge for, at tidsregistreringen kan foregå andre steder end hjemme på værksedet og derved spare spildt kørselstid.
    
    Dernæst skal projektet gøre det nemmere at bestille materialer fra deres eget værksted.

\subsubsection{Løsningsscenarier}

\subsubsection{Gevinstanalyse}

\subsubsection{Interessentanalyse}
