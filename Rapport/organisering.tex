\subsection{Organisering af Gruppearbejde}

Til at begynde med lavede gruppen en gruppekontrakt, hvor vi aftalte, hvordan vi ville holde kontakt, hvilke programmer vi vil bruge til projektforløbet.

Vi delte email, mobil nr., og blev enige om hvorledes vi ville samarbejde og kommunikere internt.

Gruppens ambitionsniveau blev også diskuteret, så alle var enige om de forventninger, som vi havde til hinanden.
Der blev også talt om, hvordan folk helst ville løse opgaverne under den konventionelle undervisning, så der ikke var nogen, der havde en forventning om, at de blev lavet fælles i gruppen, mens andre hellere ville sidde og lave det selv.

Vi aftalte også at dele filer gennem OneDrive, bruge \LaTeX til at skrive rapporten, bruge GitHub til versionsstyring af programmet og rapporten, at bruge Trello som et digitalt scrumboard og LucidChart til artefakter.

Mest ofte brugte vi Discord, et gratis chatprogram, til at kommunikere med hinanden uden for aftalte mødetider.
Det hjalp gevaldigt at vi alle blev enige om hvilke programmer vil ville bruge til at starte med, så vi undgik at skulle diskutere om det senere henne, eller at skulle have flere forskellige programmer til den samme opgave.

Endvidere lavede vi en kvalitetsplan for at sørge for kvalitetssikring.
Vi gav alle ansvar for at sikre kvalitetsvurdering af de forskellige artefakter, koden eller andet såsom at sikre projektlog udføres.
Dette sørgede for det ikke blev kun en persons ansvar at påminde gruppen, istedet hjalp vi hinanden til at huske det.

