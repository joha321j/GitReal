\section{Konklusion}
Halvvorsen ApS havde problemer med, at alt kommunikation, inklusivt timeregistrering, foregik på papir eller mundtligt. 
Dette er noget de går og døjer med, da det betyder at tømrerne skal aflevere deres timeseddeler i hånden til sekretæren.
%Det er et problem de gerne ville se løst, og vi ivrige efter at begynde projektet. 

Til projektet benyttede vi os af de kompetencer vi har fået igennem de første måneder af uddannelsen.
Både systemudviklingsartefakter og programmeringserfaring var kritiske til udviklingen af produktet. 
De førnævnte kompetencer blev brugt igennem hele projektet, men planlægningen af arbejde og udvikling foregik mere agilt med SCRUM.
Vi begik mange begynderfejl under projektet, men med fokus på positiv udvikling af vores kompetencer, blev de set til i en evaluering.
 
Løsningen til Halvvorsen ApS' problem blev igennem vores software delvist løst. 
Til dels på grund af manglende funktioner, og til dels på grund af PO's krav blev bedre formuleret senere i projektet.
Programmet kan i dens nuværende tilstand ikke tilfredstillende løse problemet PO står med, men med videreudvikling vil det kunne erstatte deres papir-timeseddeler fuldstændigt, og derved spare dem både tid og penge i processen.

Projektet har givet os erfaring med at arbejde sammen med et firma for at hjælpe dem med at analysere og løse et konkret problem.