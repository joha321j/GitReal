\section{Videreudvikling af program}

Under vores møder med virksomheden's PO og sekretæren udarbejdede vi flere forskellige PBI'er, men pga. projektets længde vidste vi fra begyndelsen at vi ikke kunne nå dem alle. Vi har således sammen med virksomheden prioriteret PBI'er, ud fra hvad de mener gav mest forretningsværdi.

Desværre nåede vi ikke at blive færdig med vores første PBI, dog fik vi lavet større dele af dem. Under demo'en af programmet fik vi meget feedback på programmet i dens form. Vi nåede dog ikke at rette op på alle de ønskede ændringer. F.eks. i det nuværende program ligger sygedage, feriedage og andet på alle timesedler, som vi efter demo'en fik at vide ikke var ønsket, og det hellere være separeret fra sedlerne. Andetvis kører programmet gennem et konsol vindue, så programmet kun kan interagere med gennem indtastelse på et keyboard. Fremtidigt kunne vi lave et GUI for en bedre interaktion mellem brugerne og programmet. Referat af demo kan findes i bilag.

I slutningen af projektet så vores PBI-listen ud således:
\begin{itemize}
\item Registerring af timer (XL)
\item Bestilling af materialer fra værkstedet (L) 
\item Udsendelse af besked (M)
\item Check af arbejdstegning (XXL)
\end{itemize}


