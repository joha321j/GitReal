\subsection{SCRUM}
Vi benyttede os af SCRUM til planlægningen og udførslen af arbejdet.
For nemmest at formidle vores brug af SCRUM, vil vi gennemgå hver af stadierne, som vi gennemgik under udviklingen af produktet. 

\subsubsection{Sprint Planning Meeting}
Et sprint planning meeting er det første skridt imod en velformuleret plan.
Under et sprint planning meeting skriver eller omskriver man sin product backlog.
Det vil sige at man koordinere med sin PO, og sit hold hvad product backloggen skal indeholde, og hvordan den skal prioriteres.
Det gjorde vores møder sværere, da vi ikke havde mulighed for at holde sprint planning meetings sammen med vores PO.
Vi valgte i stedet selv at prioriterer backloggen, og sende den forbi PO til feedback og evt. omprioritering.
Vi holdte 3 sprint planning meetings i løbet af projektet.\cite{scrum} 

Det første var til pre-sprintet, hvor vi planlagde alt der skulle vores product backlog og fyldte den til renden med product backlog items.
Yderligere planlagde vi hvilke systemudviklingsartefakter der skulle skrives før sprint 1.

Sprint 1 og 2 blev udført som først beskrevet.

\subsubsection{Daily Sprint Meeting}
Som en del af vores daglige ritual satte vi os ned i gruppen og diskuterede, hvor langt vi var med det arbejde, vi påbegyndte dagen før.
Det varierede med, hvor meget vi arbejdede i gruppen eller i par, så længden af disse møder varierede meget.
De tog dog aldrig længere end 10 minutter.
Som en del af vores daily sprint meeting planlagde vi også, hvad vi skulle nå for dagen, og hvordan arbejdsfordelingen skulle ske for dagen.

\subsubsection{Produkt Demo}
Da PO skal stille sig tilfreds med løsningen før, vi kan kalde projektet afsluttet, er det vigtigt løbende at lave demoer.\cite{scrumdemo}
Demoerne afholdes som en del af afslutningen af en sprint.
Det medfører at løsningen ikke kommmer ud på et sidespor, eller tager en retning som PO helst ser undgået, og product backlog items løbende kan omprioriteres.

Vores tilgang til demoer har været at forsøge at afholde en demo efter hver afsluttet sprint.
Vi har dog lavet en planlægningsfejl, og derfor har vi kun kunne lave én demo for PO.
Det der gik galt var, at vi ikke planlagde fornuftigt efter PO's skema.
PO gjorde det klart, at de havde tid hver onsdag, og vi var velkommende til at komme forbi.
Vores problem blev at produktet ikke var klar til en demo, den onsdag vi planlagde, at det skulle være det.
Vi havde været for optimistiske, og allokeret mere arbejde til første sprint, end hvad vi kunne levere.
Det resulterede til gengæld i en evaluering af vores pre-sprint planning meeting, og vi fik derved lært noget af processen alligevel.
   
Til mødet planlagde vi hvilke funktioner, der skulle gennemgås, samt at vi ville benytte Concurrent Think Aloud(CTA) metoden til formålet.\cite{cta}
   
For at læse hele referatet af mødet med PO se bilagene på side \pageref{demoreferat}.

Resultatet af demoen vil blive diskuteret i afsnit \ref{vidu}.