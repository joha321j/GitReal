% !TeX spellcheck = da

\subsection Objekt- og Domænemodeller
\subsubsection Objektmodellen
Halvorsen ApS er et firma med mange ansatte, og et strengt hieraki imellem dem.

Det er derfor nødvendigt at differentiere imellem jobstillinger indenfor firmaet.

Udover personalet skal der differentieres mellem forskellige byggepladser.
 
Der er på et givent tidspunkt en til flere byggepladser, og de skal alle sammen registreres til en eller flere sager, og skal så vidt adskilles. 

Det individuelle sager er adskilt med et sagsid. Et eksempel på en sag, ses på objektet "Tagmontering Etape 1". 

I den sag ser man en forbindelse til byggepladsen, hvor tagmonteringen foregår, og yderligere en forbindelse til svenden Torben, som netop er ude og montere tagene. 

Derudover er objekterne sat op med forbindelser, for hvad de har direkte kontakt med.

Her er selvfølgelig udelukket irrelevante faktorer, som fx. at de alle sammen snakker sammen til årets julefrokost. 
\subsubsection Domænemodellen
Ved et første øjekast på domænemodellen, vil man lægge mærke til, at sekretæren kun har ét forhold. Dette forhold defineres ved en én til én forbindelse til tømrermesteren. Som en af de primære brugere af programmet, kan hun derigennem få fuld adgang til resten af firmaet gennem netop tømrermesteren.